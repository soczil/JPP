\documentclass[a4paper]{article}

\usepackage[T1]{fontenc}
\usepackage[utf8]{inputenc}
\usepackage[polish]{babel}
\usepackage{a4wide}

\begin{document}

\begin{center}
    \textsc{\textbf{\LARGE JPP - Interpreter}}\\
    \textsc{\large Opis języka}\\
    \textsc{\bf Karol Soczewica (ks394468)}
\end{center}

\vspace{0.8cm}

\section*{Opis}
Język, którego interpreter będę implementował to język Latte z pewnymi dodatkami/zmianami (Latte++).
Wykonywanie programu będzie zaczynało się od funkcji $\textbf{int main() \{\}}$, która w programie
musi wystąpić.\\
W Latte++ dodatkowym typem będzie jednowymiarowa tablica.\\ 
Dodatkami do instrukcji Latte są instrukcje $\textbf{for}$, $\textbf{foreach}$ (tak jak w Javie), 
$\textbf{break}$, $\textbf{continue}$ oraz $\textbf{print}$. Oprócz tego język Latte++ będzie miał
słowo kluczowe $\textbf{final}$, dzięki któremu będzie można mieć zmienne tylko do odczytu,
a intrukcja $\textbf{if}$ zostanie rozszerzona o możliwość dodawania bloków $\textbf{else if}$.\\
Zmianą w składni w stosunku do Latte jest to, że instrukcje w pętlach oraz w $\textbf{if}$ muszą
zawsze znajdować się w środku bloku, czyli nielegalne będzie napisanie $\textbf{if (i < 5) print(i)}$,
zamiast tego trzeba będzie napisać $\textbf{if (i < 5) \{ print(i) \}}$.\\
Każda zmienna musi być zadeklarowana przed użyciem. Jeśli zmienna nie jest zainicjalizowana, to
przyjmuje domyślną dla swojego typu wartość: \textbf{int} $\rightarrow$ 0, \textbf{bool} $\rightarrow$ false,
\textbf{string} $\rightarrow$ $"$$"$.\\
Zmienne deklarowane wewnątrz bloków przesłaniają zmienne o tych samych nazwach spoza bloku.\\
Parametry funkcji przekazywane są przez wartość.

\section*{Gramatyka}
-- Programs --------------------------------------------\\

entrypoints Program ;

Program. Program ::= [FunDef] ;
\\\\
\noindent
-- Types ------------------------------------------------\\

Int. Type ::= ``int'' ;

Str. Type ::= ``string'' ;

Bool. Type ::= ``bool'' ;

Void. Type ::= ``void'' ;

Array. Type ::= ``Array'' ``<'' Type ``>'' ;
\\\\
\noindent
-- Expressions ---------------------------------------\\

Evar. Expr6 ::= Ident ;

ELitInt. Expr6 ::= Integer ;

ELitTrue. Expr6 ::= ``true'' ;

ELitFalse. Expr6 ::= ``false'' ;

EApp. Expr6 ::= Ident ``('' [Expr] ``)'' ;

EString. Expr6 ::= String ;

Neg. Expr5 ::= ``-'' Expr6 ;

Not. Expr5 ::= ``!'' Expr6 ;

EMul. Expr4 ::= Expr4 MulOp Expr5 ;

EAdd. Expr3 ::= Expr3 AddOp Expr4 ;

ERel. Expr2 ::= Expr2 RelOp Expr3 ;

EAnd. Expr1 ::= Expr1 ``\&\&'' Expr2 ;

EOr. Expr ::= Expr ``||'' Expr1 ;

coercions Expr 6 ;

separator Expr ``,'' ;
\\\\
\noindent
-- Operators --------------------------------------------\\

Plus. AddOp ::= ``+'' ;

Minus. AddOp ::= ``-'' ;

Times. MulOp ::= ``*'' ;

Div. MulOp ::= ``/'' ;

Mod. MulOp ::= ``\%'' ;

Lt. RelOp ::= ``<'' ;

Leq. RelOp ::= ``<='' ;

Gt. RelOp ::= ``>'' ;

Geq. RelOp ::= ``>='' ;

Eq. RelOp ::= ``=='' ;

Neq. RelOp ::= ``!='' ;
\\\\
\noindent
-- Comments -------------------------------------------\\

comment ``\#'' ;

comment ``//'' ;

comment ``/*'' ``*/'' ;


\end{document}