\documentclass[a4paper]{article}

\usepackage[T1]{fontenc}
\usepackage[utf8]{inputenc}
\usepackage[polish]{babel}
\usepackage{a4wide}
\usepackage{listings}
\usepackage{color}

\definecolor{dkgreen}{rgb}{0,0.6,0}
\definecolor{gray}{rgb}{0.5,0.5,0.5}
\definecolor{mauve}{rgb}{0.58,0,0.82}

\lstset{frame=tb,
  language=Java,
  aboveskip=3mm,
  belowskip=3mm,
  showstringspaces=false,
  columns=flexible,
  basicstyle={\small\ttfamily},
  numbers=none,
  numberstyle=\tiny\color{gray},
  keywordstyle=\color{blue},
  commentstyle=\color{dkgreen},
  stringstyle=\color{mauve},
  breaklines=true,
  breakatwhitespace=true,
  tabsize=3,
  escapeinside={\%}{)}
}

\begin{document}

\begin{center}
    \textsc{\textbf{\LARGE JPP - Interpreter}}\\
    \textsc{\large Opis języka}\\
    \textsc{\bf Karol Soczewica (ks394468)}
\end{center}

\vspace{0.8cm}

\section*{Opis}
Język, którego interpreter będę implementował to język Latte z pewnymi dodatkami/zmianami (Latte++).
Wykonywanie programu będzie zaczynało się od funkcji $\textbf{int main() \{\}}$, która w programie
musi wystąpić.\\
W Latte++ dodatkowym typem będzie jednowymiarowa tablica.\\ 
Dodatkami do instrukcji Latte są instrukcje $\textbf{for}$, $\textbf{foreach}$ (tak jak w Javie), 
$\textbf{break}$, $\textbf{continue}$ oraz $\textbf{print}$. Oprócz tego język Latte++ będzie miał
słowo kluczowe $\textbf{final}$, dzięki któremu będzie można mieć zmienne tylko do odczytu.
Intrukcja $\textbf{if}$ zostanie rozszerzona o możliwość dodawania bloków $\textbf{else if}$.\\
Zmianą w składni w stosunku do Latte jest to, że instrukcje w pętlach oraz w $\textbf{if}$ muszą
zawsze znajdować się w środku bloku, czyli nielegalne będzie napisanie $\textbf{if (i < 5) print(i);}$,
zamiast tego trzeba będzie napisać $\textbf{if (i < 5) \{ print(i); \}}$.\\
Każda zmienna musi być zadeklarowana przed użyciem. Jeśli zmienna nie jest zainicjalizowana, to
przyjmuje domyślną dla swojego typu wartość: \textbf{int} $\rightarrow$ 0, \textbf{bool} $\rightarrow$ false,
\textbf{string} $\rightarrow$ $"$$"$.\\
Zmienne deklarowane wewnątrz bloków przesłaniają zmienne o tych samych nazwach spoza bloku.\\
Parametry funkcji przekazywane są przez wartość.

\section*{Gramatyka}
Gramatyka LBNF:\\
-- Programs --------------------------------------------\\

entrypoints Program ;

Program. Program ::= [FunDef] ;
\\\\
\noindent
-- Types ------------------------------------------------\\

Int. Type ::= ``int'' ;

Str. Type ::= ``string'' ;

Bool. Type ::= ``bool'' ;

Void. Type ::= ``void'' ;

Array. Type ::= ``Array'' ``<'' Type ``>'' ;
\\\\
\noindent
-- Statements ------------------------------------------\\

Block. Block ::= ``\{'' [Stmt] ``\}'' ;

separator Stmt ``'' ;

BStmt. Stmt ::= Block ;

Empty. Stmt ::= ``;'' ;

FunDef. FunDef ::= Type Ident ``('' [Arg] ``)'' Block ;

separator nonempty FunDef ``'' ;

Arg. Arg ::= Type Ident ;

separator Arg ``,'' ;

FStmt. Stmt ::= FunDef ;

ArrDecl. Stmt ::= ``Array'' ``<'' Type ``>'' Ident ``='' Expr ``**'' ``['' Expr ``]'' ``;'' ;

ArrAss. Stmt ::= Ident ``['' Expr ``]'' ``='' Expr ``;'' ;

DStmt. Stmt ::= Decl ;

NormalDecl. Decl ::= Type [Item] ``;'' ;

FinalDecl. Decl ::= ``final'' Type [Item] ``;'' ;

NoInit. Item ::= Ident ;

Init. Item ::= Ident ``='' Expr ;

separator nonempty Item ``,'' ;

Ass. Stmt ::= Ident ``='' Expr ``;'' ;

Inc. Stmt ::= Ident ``++'' ``;'' ;

Dec. Stmt ::= Ident ``--'' ``;'' ;

Ret. Stmt ::= ``return'' Expr ``;'' ;

RetV. Stmt ::= ``return'' ``;'' ;

Break. Stmt ::= ``break'' ``;'' ;

Continue. Stmt ::= ``continue'' ``;'' ;

Cond. Stmt ::= ``if'' ``('' Expr ``)'' Block [ElseIf] ;

CondElse. Stmt ::= ``if'' ``('' Expr ``)'' Block [ElseIf] ``else'' Block ;

ElseIf. ElseIf ::= ``else if'' ``('' Expr ``)'' Block ;

separator ElseIf ``'' ;

While. Stmt ::= ``while'' ``('' Expr ``)'' Block ;

For. Stmt ::= ``for'' ``('' ForInit [Expr] ``;'' [Expr] ``)'' Block ;

ForInitExpr. ForInit ::= [Expr] ``;'' ;

ForInitVar. ForInit ::= Decl ;

ForIn. Stmt ::= ``for'' ``('' Ident ``:'' Ident ``)'' Block ; 

EStmt. Stmt ::= Expr ``;'' ;

Print. Stmt ::= ``print'' ``('' Expr ``)'' ``;'' ;
\\\\
\noindent
-- Expressions ---------------------------------------\\

Evar. Expr6 ::= Ident ;

ELitInt. Expr6 ::= Integer ;

ELitTrue. Expr6 ::= ``true'' ;

ELitFalse. Expr6 ::= ``false'' ;

EApp. Expr6 ::= Ident ``('' [Expr] ``)'' ;

EString. Expr6 ::= String ;

Neg. Expr5 ::= ``-'' Expr6 ;

Not. Expr5 ::= ``!'' Expr6 ;

EMul. Expr4 ::= Expr4 MulOp Expr5 ;

EAdd. Expr3 ::= Expr3 AddOp Expr4 ;

ERel. Expr2 ::= Expr2 RelOp Expr3 ;

EAnd. Expr1 ::= Expr1 ``\&\&'' Expr2 ;

EOr. Expr ::= Expr ``||'' Expr1 ;

coercions Expr 6 ;

separator Expr ``,'' ;
\\\\
\noindent
-- Operators --------------------------------------------\\

Plus. AddOp ::= ``+'' ;

Minus. AddOp ::= ``-'' ;

Times. MulOp ::= ``*'' ;

Div. MulOp ::= ``/'' ;

Mod. MulOp ::= ``\%'' ;

Lt. RelOp ::= ``<'' ;

Leq. RelOp ::= ``<='' ;

Gt. RelOp ::= ``>'' ;

Geq. RelOp ::= ``>='' ;

Eq. RelOp ::= ``=='' ;

Neq. RelOp ::= ``!='' ;
\\\\
\noindent
-- Comments --------------------------------------------\\

comment ``\#'' ;

comment ``//'' ;

comment ``/*'' ``*/'' ;

\section*{Przykłady}
\begin{lstlisting}
// PrintArrayElements.lpp (wypisywanie wartosci z tablicy na trzy sposoby)
int main() {
    Array<int> xs = 5 ** [1]; // utworzenie tablicy postaci [1, 1, 1, 1, 1]
    for (x : xs) {
        print(x);
    }

    for (int i = 0; i < 5; i++) {
        print(xs[i]);
    }

    int i = 0;
    while (i < 5) {
        print(xs[i]);
    }
    return 0;
}
\end{lstlisting}

\begin{lstlisting}
// PrintEvenNumbers.lpp
int main() {
    bool isEven(int x) {
        if (x % 2 == 0) {
            return true;
        } else {
            return false;
        } 
    }

    Array<int> numbers = 10 ** [0];
    for (int i = 1; i <= 10; i++) {
        numbers[i - 1] = i;
    }

    for (x : numbers) {
        if (isEven(x)) {
            print(x);
        }
    }
    return 0;
}
\end{lstlisting}

\newpage
\begin{lstlisting}
// Fib.lpp (obliczanie n-tej liczby Fibonacciego na trzy sposoby)
int fib_rec(int n) {
    if (n <= 1) {
        return n;
    }
    return fib_rec(n - 1) + fib_rec(n - 2);
}

int main() {
    int n = 10;
    print(fib_rec(n));
    print(fib_arr(n));
    print(fib_opt(n));
    return 0;
}

int fib_arr(int n) {
    Array<int> f = (n + 2) ** [0];
    int i = 2;

    f[1] = 1;
    while (i <= n) {
        f[i] = f[i - 1] + f[i - 2];
        i++;
    }

    return f[n];
}

int fib_opt(int n) {
    int f1 = 0, f2 = 1, res;
    if (n == 0) {
        return f1;
    }

    for (int i = 2; i <= n; i++) {
        res = f1 + f2;
        f1 = f2;
        f2 = res;
    }

    return f2;
}
\end{lstlisting}

\end{document}